\documentclass[aspectratio=169]{beamer}
\usepackage[utf8]{inputenc}
\usepackage[T1]{fontenc}
\usepackage{hyperref}
\usepackage{booktabs}
\usepackage{graphicx}
\usepackage{caption}
\usepackage{subcaption}
\usepackage{float}
\usepackage{multirow}
\usepackage{subcaption}
\usepackage{algorithm}
\usepackage[noend]{algpseudocode}
\usepackage{amssymb}
\usepackage{amsmath}
\usepackage{pifont}
\usepackage{cleveref}
\usepackage{mathabx}
\usepackage{mathpazo}
\usepackage{eulervm}
\usepackage{natbib}
\usepackage{listings}
\usepackage{color}
\usepackage[export]{adjustbox}
\usepackage{appendixnumberbeamer}
\usetheme{pro}
\newcommand{\cmark}{\ding{51}}
\newcommand{\xmark}{\ding{55}}
\newcommand{\eop}{\vspace{10px}}
\renewcommand{\sf}[1]{\textsf{\textup{#1}}}  
\renewcommand{\tt}[1]{\texttt{\textup{#1}}}  
\definecolor{mygreen}{rgb}{0,0.6,0}
\definecolor{mygray}{rgb}{0.5,0.5,0.5}
\definecolor{mymauve}{rgb}{0.58,0,0.82}
\newcommand{\g}[1]{{\color{green}#1}}
\renewcommand{\b}[1]{{\color{blue}#1}}
\renewcommand{\r}[1]{{\color{red}#1}}  
\lstset{ 
  % backgroundcolor=\color{white},   % choose the background color; you must add \usepackage{color} or \usepackage{xcolor}; should come as last argument
  basicstyle=\footnotesize,        % the size of the fonts that are used for the code
  breakatwhitespace=false,         % sets if automatic breaks should only happen at whitespace
  breaklines=true,                 % sets automatic line breaking
  captionpos=b,                    % sets the caption-position to bottom
  commentstyle=\color{mygreen},    % comment style
  deletekeywords={...},            % if you want to delete keywords from the given language
  escapeinside={\%*}{*)},          % if you want to add LaTeX within your code
  extendedchars=true,              % lets you use non-ASCII characters; for 8-bits encodings only, does not work with UTF-8
  firstnumber=1,                   % start line enumeration with line 1000
  frame=single,	                   % adds a frame around the code
  keepspaces=true,                 % keeps spaces in text, useful for keeping indentation of code (possibly needs columns=flexible)
  keywordstyle=\color{blue},       % keyword style
  language=Octave,                 % the language of the code
  morekeywords={*,...},            % if you want to add more keywords to the set
  numbers=left,                    % where to put the line-numbers; possible values are (none, left, right)
  numbersep=5pt,                   % how far the line-numbers are from the code
  numberstyle=\tiny\color{mygray}, % the style that is used for the line-numbers
  rulecolor=\color{black},         % if not set, the frame-color may be changed on line-breaks within not-black text (e.g. comments (green here))
  showspaces=false,                % show spaces everywhere adding particular underscores; it overrides 'showstringspaces'
  showstringspaces=false,          % underline spaces within strings only
  showtabs=false,                  % show tabs within strings adding particular underscores
  stepnumber=1,                    % the step between two line-numbers. If it's 1, each line will be numbered
  stringstyle=\color{mymauve},     % string literal style
  % tabsize=2,	                   % sets default tabsize to 2 spaces
  % title=\lstname                 % show the filename of files included with \lstinputlisting; also try caption instead of title
}
%% Load the markdown package
\usepackage[citations,footnotes,definitionLists,hashEnumerators,smartEllipses,tightLists=false,pipeTables,tableCaptions,hybrid,fencedCode]{markdown}
%%begin novalidate
\markdownSetup{rendererPrototypes={
    link = {\href{#2}{#1}},
    headingOne = {\section{#1}},
    headingTwo = {\subsection{#1}},
    headingThree = {\begin{frame}\frametitle{#1}},
    headingFour = {\begin{block}{#1}},
    horizontalRule = {\end{block}}
}}
%%end novalidate
\begin{document}
\title[Footnote Title]{Title}
\date[DOLAP]{DOLAP2020}
\author[Matteo Francia (UniBO)]{
    \textit{\textbf{Matteo Francia}}\inst{1}\\
    \url{m.francia@unibo.it}
}
\institute[UniBO]{\inst{1} University of Bologna}

\begin{frame}
\titlepage
\end{frame}

\frame{\tableofcontents}

\begin{markdown}
%%begin novalidate

# How?

### We can do that?

- Yeah to some \g{extent}, \b{with} \texttt{markdown} package :-)
    - __$\hash$__ and __$\hash\hash$__ for section and subsection headers (in ToC)
    - Redefine __$\hash\hash\hash$__ to start a frame and frametitle
    - (Nested) bullet and numbered lists
    - Text formatting (*italic*, **bold becomes italic + alerted**) 
    - Redefine __$\hash\hash\hash\hash$__ to start a block with title \linebreak
      and __\texttt{-{}-{}-{}-}__ to end the block
    - ___Compile with \texttt{-{}-shell-escape}___ (Overleaf does this already)
- (Alternative approaches: Pandoc, wikitobeamer)

\end{frame}

### This is code

``` Java
public class Foo { ... }
```

``` Python
def main():
    ...
```

\setkeys{Gin}{width=.5\linewidth}
![img](imgs/25550.jpg)
\end{frame}

%%%%%%%%%%%%%%%%%%%%%%

### Caveats

- Nothing too complicated! 
- No verbatim or fragile stuff!
- No $\hash$ and \textunderscore{} characters!\linebreak 
  (I used `$\hash$` and `\textunderscore`)
- Can't pass options to frames
- __Need to write \texttt{\textbackslash end\string{frame\string}} manually!__

\end{frame}

%%%%%%%%%%%%%%%%%%%%%%


%%% # and ## can still be used as sections and subsections if you prefer
# Example

## Proposed Menus

%%% ### starts a frame + frametitle
### Breakfast Menu

%%% bulleted lists as usual 

- Eggs
    * scrambled
    * sunny-side-up
- Coffee
    * Americano
    * Long black
- Tea
    * Darjeeling
    * English Breakfast

%%% Due to the complicatedness of beamer frames, \end{frame} MUST appear in the source code itself and cannot be "hidden" in another command

\end{frame}

%%%%%%%%%%%

### Lunch Menu

- Spaghetti
    * Bolognese
    * Aglio olio
- Sandwiches
    * Egg
    * Ham
    * Tuna

\end{frame}

%%%%%%%%%%%

## Budgeting

### Projected Profit

1. And the answer is...
2. $f(x)=\sum_{n=0}^\infty\frac{f^{(n)}(a)}{n!}(x-a)^n$
    #. How do we _know_ that?
    #. __Maths!__

\end{frame}

### Testing blocks

#### This is a block!

- Here is some content.
- Here's more contents.

---

\end{frame}


### Citations

- This is a citation [@DBLP:journals/snam/FranciaGG19]

\end{frame}


### Pipe Tables

- Use `pipeTables` and `tableCaptions` options
- Available since `markdown` v2.8.0

| Right | Left | Default | Center |
|------:|:-----|---------|:------:| 
|  12   |  12  |  12     |   12   | 
| 123   |  123 |   123   |  123   | 
|   1   |    1 |     1   |    1   | 

  : Demonstration of pipe table syntax.
  
\end{frame}

%%novalidate
\end{markdown}


{
\bgimg{0.3}{imgs/25550.jpg}
\begin{frame}{A longer title}
    \begin{itemize}
    \item one
    \item two
    \end{itemize}

This is a citation \cite{DBLP:journals/snam/FranciaGG19}
\end{frame}
}

{
\setbeamercolor{background canvas}{bg=UBCblue,fg=white}
\begin{frame}
    \centering
    \topskip0pt
    \vspace*{\fill}
    \begin{color}{white}\Huge End \end{color} %Questions?
    \vspace*{\fill}
\end{frame}
}

% \appendix

% \begin{frame}
% \renewcommand{\bibfont}{\footnotesize}
% \frametitle{Bibliography}
% \bibliographystyle{apalike}
% \bibliography{refs}
% \end{frame}

\begin{frame}[allowframebreaks]{References}
    \tiny
    \bibliographystyle{unsrt}
    \bibliography{refs}
\end{frame}

\end{document}